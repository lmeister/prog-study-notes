% chapters/chapter1.tex
\documentclass[../main.tex]{subfiles}

\begin{document}
	
	\section{Datenstrukturen}
%	   TEMPLATE
%		\subsection{Titel}
%		- GENERAL INFORMATION -\\\\
%		\textbf{Anwendungsfälle:}\\
%		- Punkt 1\\
%		- Punkt 2\\
%		\\
%		\textbf{Syntax:}
%		\lstinputlisting[language=Java,caption=Example.java]{Example.java} 
%		\\
%		\textbf{Funktionalität unter der Haube:}\\
%		- Punkt 1\\
%		- Punkt 2\\
%		\clearpage
	
		\subsection{ArrayList}
		Die ArrayList in Java ist eine dynamisch wachsende Liste, die auf dem zugrunde liegenden Array basiert.
		Sie erweitert die Funktionalität von Arrays durch automatische Größenanpassung und bietet eine flexible Möglichkeit, Elemente hinzuzufügen, zu löschen und darauf zuzugreifen. \\\\
		\textbf{Anwendungsfälle:}\\
		- Speicherung von Elementen einer Liste\\
		- Zugriff auf Elemente durch Index\\
		- Wenn die Größe der Liste nicht im Voraus bekannt ist\\ 
		- Wenn häufiger Einfüge- und Löschoperationen erforderlich sind\\\\
		\textbf{Syntax:}
		\lstinputlisting[language=Java,caption=ArrayList.java]{code/ArrayList.java} 
		\\
		\textbf{Funktionalität unter der Haube:}\\
		- Intern durch Array verwaltet, welches ggf. neu dimensioniert werden muss\\
		- Neudimensionierung führt zu O(n), da neues Array und alle Elemente kopiert werden. Wird allerdings selten gemacht und dann i.d.R. direkt verdoppelt, somit on average konstant (O(n))\\\\
		\textbf{Laufzeitkomplexität}\\
		\begin{table}[ht]
			\centering
			\begin{tabular}{l *{6}{c}}
				\toprule
				Operation & Add & Remove & Get & Contains & Next & Size \\
				\midrule
				Time Complexity & $O(1)$ & $O(n)$ & $O(1)$ & $O(n)$ & $O(1)$ & $O(1)$ \\
				\bottomrule
			\end{tabular}
			\label{tab:arraylist-complexities}
		\end{table}
		\clearpage
		
		\subsection{Titel}
		- GENERAL INFORMATION -\\\\
		\textbf{Anwendungsfälle:}\\
		- Punkt 1\\
		- Punkt 2\\
		\\
		\textbf{Syntax:}
		\lstinputlisting[language=Java,caption=Example.java]{Example.java} 
		\\
		\textbf{Funktionalität unter der Haube:}\\
		- Punkt 1\\
		- Punkt 2\\\\
		
		
		\clearpage

\end{document}